\documentclass{article}
\usepackage{minted}
\usepackage{xcolor} % Agrega el paquete xcolor para la personalización de colores
\usepackage{color} % Agrega el paquete xcolor para la personalización de colores
\usepackage{keyval} % Agrega el paquete xcolor para la personalización de colores

% Define un nuevo estilo para Java con colores personalizados
\definecolor{customcolor}{RGB}{60, 65, 60} % Color de fondo similar al de VSCode
\definecolor{codecomment}{RGB}{86,156,214} % Color de comentarios en VSCode
\definecolor{codekeyword}{RGB}{198,120,221} % Color de palabras clave en VSCode
\definecolor{codestring}{RGB}{209,105,105} % Color de cadenas en VSCode

\newminted{java}{
  linenos,
  frame=lines,
  framesep=2mm,
  baselinestretch=1.2,
  fontsize=\footnotesize,
  tabsize=4,
  breaklines,
  escapeinside=||,
  mathescape,
  % Configuración de colores personalizados
  bgcolor=customcolor,
  style=vim, % Puedes cambiar el estilo si lo prefieres
  %commentstyle=\color{codecomment}, % Comentarios
  %keywordstyle=\color{codekeyword}, % Palabras clave
  %stringstyle=\color{codestring}, % Cadenas
  numbersep=5pt,
  rulecolor=codeframe,
}

\begin{document}

\section{Ejemplo de Código Java}

A continuación se muestra un ejemplo de código Java con una configuración que se asemeja a Visual Studio Code:

\begin{javacode}
import java.util.*;

public class HelloWorld {
    public static void main(String[] args) {
        // Este es un comentario
        Scanner sc = new Scanner(System.in);
        String h = sc.nextLine();
        int i = 0;
        System.out.println("Hello, World!");
        
    }
}
\end{javacode}

\end{document}
